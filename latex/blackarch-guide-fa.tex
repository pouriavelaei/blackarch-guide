%%%%%%%%%%%%%%%%%%%%%%%%%%%%%%%%%%%%%%%%%%%%%%%%%%%%%%%%%%%%%%%%%%%%%%%%%%%%%%%%
%                                                                              %
% BlackArch Linux Guide                                                        %
%                                                                              %
%%%%%%%%%%%%%%%%%%%%%%%%%%%%%%%%%%%%%%%%%%%%%%%%%%%%%%%%%%%%%%%%%%%%%%%%%%%%%%%%

\documentclass[a4paper, oneside, 11pt]{book}

%%% INCLUDES %%%
\renewcommand{\familydefault}{\sfdefault}

\usepackage{array}
\usepackage{color}
\usepackage{enumerate}
\usepackage{fancyhdr}
\usepackage{fancyvrb}
\usepackage{geometry}
\usepackage{graphicx}
\usepackage{html}
\usepackage{hyperref}
\usepackage{ifpdf}
\usepackage{listings}
\usepackage{pstricks}
\usepackage{supertabular}
\usepackage{tocloft}
\usepackage[utf8]{inputenc}

%%% LAYOUT %%%
\setlength{\parindent}{0em}
\setlength{\parskip}{1.5ex plus0.5ex minus0.5ex}
\geometry{left=2.5cm, textwidth=16cm, top=3cm, textheight=25cm, bottom=3cm}
\widowpenalty=2000
\clubpenalty=1000
\frenchspacing
\sloppy
\pagecolor[HTML]{FFFFFF}
\color[HTML]{333333}
\setcounter{tocdepth}{10}
\setcounter{secnumdepth}{10}

\hypersetup{
  pdftitle={BlackArch Linux, The BlackArch Linux Guide},
  pdfsubject={BlackArch Linux, The BlackArch Linux Guide},
  pdfauthor={BlackArch Linux, BlackArch Linux},
  pdfkeywords={BlackArch Linux, Penetration Testing, Security, Hacking, Linux},
  pdfcenterwindow=true,
  colorlinks=true,
  breaklinks=true,
  linkcolor=blue,
  menucolor=blue,
  urlcolor=blue
}

% syntax highlighting
% all options should be set here document wide
\lstset{
backgroundcolor=\color[HTML]{EEEEEE},
frame=single,
basicstyle=\footnotesize\ttfamily,
float,
deletekeywords={return},
otherkeywords={mkdir, curl, sudo, sha1sum, grep, cut, sort, wget, makepkg,
pacman, blackman, chmod},
keywordstyle=\color{orange},
commentstyle=\color{blue},
stringstyle=\color{red},
language=bash,
showspaces=false,
showtabs=false,
tabsize=2
}

%%% HEADER / FOOTER %%%
\setlength{\headheight}{33pt}
\setlength{\headsep}{33pt}
\lhead{{\includegraphics[width=1cm,height=1cm]{images/logo.png}}}
\rhead{راهنمای BlackArch Linux}

%%% CUSTOM MACROS %%%
% for html links
\ifpdf\else
\def\href#1#2{\htmladdnormallink{#2}{#1}}
\fi

%------------------%
%  TITLE PAGE      %
%------------------%
\begin{document}
\pagestyle{empty}
\begin{center}
\begin{figure}[htbp]
\centering
\vspace{0.5cm}
\includegraphics[width=8cm]{images/logo.png}
\label{fig:logo}
\end{figure}
\vspace{0.5cm}
\Huge{\textbf{راهنمای BlackArch Linux}}\\
\vspace{1cm}
\Large{\color{blue}https://www.blackarch.org/}\\
\vspace{0.5cm}
\end{center}
\newpage
\tableofcontents
\newpage
\pagestyle{fancy}

%------------------%
%  Chapter 1       %
%------------------%

\chapter{مقدمه}

\section{نمای کلی}
راهنمای BlackArch Linux به چند بخش تقسیم شده است:
\begin{itemize}
\item مقدمه - نمای کلی گسترده، معرفی و اطلاعات مفید اضافی پروژه را ارائه می‌دهد
\item راهنمای کاربر - همه چیزهایی که یک کاربر معمولی برای استفاده مؤثر از BlackArch نیاز دارد بداند
\item راهنمای توسعه‌دهنده - چگونه شروع به توسعه و مشارکت در BlackArch کنیم
\item راهنمای ابزارها - جزئیات عمیق ابزارها به همراه نمونه‌های استفاده (در حال کار)
\end{itemize}

\section{BlackArch Linux چیست؟}
BlackArch یک توزیع کامل لینوکس برای تست‌کنندگان نفوذ و محققان امنیت است.
از \href{https://www.archlinux.org/}{ArchLinux} مشتق شده و کاربران می‌توانند اجزای BlackArch را
به صورت جداگانه یا در گروه‌ها مستقیماً روی آن نصب کنند.

مجموعه ابزارها به عنوان یک
\href{https://wiki.archlinux.org/index.php/Unofficial\_User\_Repositories}
{مخزن کاربر غیررسمی} Arch Linux توزیع می‌شود بنابراین شما می‌توانید BlackArch را روی
نصب موجود Arch Linux نصب کنید. بسته‌ها می‌توانند به صورت جداگانه یا بر اساس
دسته نصب شوند.

مخزن در حال گسترش دائمی در حال حاضر شامل بیش از \href{https://www.blackarch.org/tools.html}{2600} ابزار است.
همه ابزارها قبل از اضافه شدن به کدبیس به طور کامل تست می‌شوند تا کیفیت مخزن حفظ شود.
% should quickly describe the testing methods/code review procedures etc

\section{تاریخچه BlackArch Linux}
به زودی...

\section{پلتفرم‌های پشتیبانی شده}
به زودی...

\section{مشارکت کنید}
شما می‌توانید با تیم BlackArch از طریق راه‌های زیر در تماس باشید:

وب‌سایت: \url{https://www.blackarch.org/}

ایمیل: \href{mailto:team@blackarch.org}{team@blackarch.org}

IRC: \url{irc://irc.freenode.net/blackarch}

توییتر: \url{https://twitter.com/blackarchlinux}

گیت‌هاب: \url{https://github.com/Blackarch/}

دیسکورد: \url{https://discord.com/invite/xMHt8dW}

%------------------%
%  Chapter 2       %
%------------------%


\chapter{راهنمای کاربر}

\section{نصب}
بخش‌های زیر به شما نشان می‌دهند که چگونه مخزن BlackArch را راه‌اندازی کنید و
بسته‌ها را نصب کنید. BlackArch هم از نصب از مخزن با استفاده از
بسته‌های باینری و هم کامپایل و نصب از منابع پشتیبانی می‌کند.

BlackArch با نصب‌های عادی Arch سازگار است. به عنوان یک
مخزن کاربر غیررسمی عمل می‌کند. اگر بجای آن یک ISO می‌خواهید، بخش
\href{https://www.blackarch.org/downloads.html#iso}{ISO ها} را ببینید.

\subsection{نصب روی ArchLinux}
\href{https://blackarch.org/strap.sh}{strap.sh} را به عنوان root اجرا کنید و
دستورالعمل‌ها را دنبال کنید. مثال زیر را ببینید.
\begin{lstlisting}
   curl -O https://blackarch.org/strap.sh
   sha1sum strap.sh # should match: 5ea40d49ecd14c2e024deecf90605426db97ea0c
   sudo chmod +x strap.sh
   sudo ./strap.sh
\end{lstlisting}

حالا یک کپی تازه از لیست بسته‌های اصلی دانلود کنید و بسته‌ها را همگام‌سازی کنید:
\begin{lstlisting}
  sudo pacman -Syyu
\end{lstlisting}


\subsection{نصب بسته‌ها}
حالا می‌توانید ابزارهایی از مخزن blackarch نصب کنید.
\begin{enumerate}
\item برای لیست کردن همه ابزارهای موجود، اجرا کنید
\begin{lstlisting}
  pacman -Sgg | grep blackarch | cut -d' ' -f2 | sort -u
\end{lstlisting}

\item برای نصب همه ابزارها، اجرا کنید
\begin{lstlisting}
  pacman -S blackarch
\end{lstlisting}

\item برای نصب یک دسته از ابزارها، اجرا کنید
\begin{lstlisting}
  pacman -S blackarch-<category>
\end{lstlisting}

\item برای دیدن دسته‌بندی‌های blackarch، اجرا کنید
\begin{lstlisting}
  pacman -Sg | grep blackarch
\end{lstlisting}

\end{enumerate}

\subsection{نصب بسته‌ها از منبع}
به عنوان بخشی از روش جایگزین نصب، شما می‌توانید بسته‌های BlackArch را
از منبع بسازید. شما می‌توانید PKGBUILD ها را در
\href{https://github.com/BlackArch/blackarch/tree/master/packages}{github} پیدا کنید. برای
ساخت کل مخزن، می‌توانید از
ابزار \href{https://github.com/BlackArch/blackman}{Blackman} استفاده کنید.
\begin{itemize}
\item ابتدا، باید Blackman را نصب کنید. اگر مخزن بسته BlackArch
روی دستگاه شما راه‌اندازی شده است، می‌توانید Blackman را نصب کنید:
\begin{lstlisting}
  pacman -S blackman
\end{lstlisting}

\item می‌توانید Blackman را از منبع بسازید و نصب کنید:
\begin{lstlisting}
  mkdir blackman
  cd blackman
  wget https://raw.github.com/BlackArch/blackarch/master/packages/blackman/PKGBUILD
  # Make sure the PKGBUILD has not been maliciously tampered with.
  makepkg -s
\end{lstlisting}

\item یا می‌توانید Blackman را از AUR نصب کنید:
\begin{lstlisting}
  <whatever AUR helper you use> -S blackman
\end{lstlisting}

\end{itemize}

\subsection{استفاده اساسی از Blackman} Blackman استفاده از آن بسیار ساده است، هرچند فلگ‌ها با آنچه شما
معمولاً از چیزی مانند pacman انتظار دارید متفاوت است. استفاده اساسی در زیر شرح داده شده است.
\begin{itemize}
\item دانلود، کامپایل و نصب بسته‌ها:
\begin{lstlisting}
  sudo blackman -i package
\end{lstlisting}

\item دانلود، کامپایل و نصب کل دسته:
\begin{lstlisting}
  sudo blackman -g group
\end{lstlisting}

\item دانلود، کامپایل و نصب همه ابزارهای BlackArch:
\begin{lstlisting}
  sudo blackman -a
\end{lstlisting}

\item برای لیست کردن دسته‌بندی‌های blackarch:
\begin{lstlisting}
  blackman -l
\end{lstlisting}

\item برای لیست کردن ابزارهای دسته:
\begin{lstlisting}
  blackman -p category
\end{lstlisting}

\end{itemize}

\subsection{نصب از ISO کامل، netinstall یا ArchLinux}
می‌توانید BlackArch Linux را از یکی از ISO های کامل یا netinstall ما نصب کنید.\\
\url{https://www.blackarch.org/download.html#iso} را ببینید. مراحل زیر پس از
بوت ISO لازم است.
\begin{itemize}
\item بسته blackarch-installer را نصب کنید:
\begin{lstlisting}
  sudo pacman -S blackarch-installer
\end{lstlisting}

\item اجرا کنید
\begin{lstlisting}
  sudo blackarch-install
\end{lstlisting}

\end{itemize}

%------------------%
%  Chapter 3       %
%------------------%

\chapter{راهنمای توسعه‌دهنده}

\section{سیستم ساخت Arch و مخازن}

فایل‌های PKGBUILD اسکریپت‌های ساخت هستند. هر کدام به makepkg(1) می‌گوید که چگونه یک
بسته بسازد. فایل‌های PKGBUILD در Bash نوشته می‌شوند.

برای اطلاعات بیشتر، موارد زیر را بخوانید (یا مرور کنید):
\begin{itemize}
\item \href{https://wiki.archlinux.org/index.php/Creating_Packages}{Arch Wiki: ایجاد بسته‌ها}
\item \href{https://wiki.archlinux.org/index.php/Makepkg}{Arch Wiki: makepkg}
\item \href{https://wiki.archlinux.org/index.php/PKGBUILD}{Arch Wiki: PKGBUILD}
\item \href{https://wiki.archlinux.org/index.php/Arch_Packaging_Standards}{Arch Wiki: استانداردهای بسته‌بندی Arch}
\end{itemize}

\section{استانداردهای PKGBUILD Blackarch}
برای سادگی، PKGBUILD های ما شبیه به آن‌های AUR هستند،
با چند تفاوت کوچک که در زیر شرح داده شده است. هر بسته باید
حداقل متعلق به blackarch باشد، همچنین تداخل زیادی وجود خواهد داشت با
چندین بسته متعلق به چندین گروه.

\subsection{گروه‌ها}
برای اینکه کاربران بتوانند طیف خاصی از بسته‌ها را سریع و آسان نصب کنند،
بسته‌ها به گروه‌ها تقسیم شده‌اند. گروه‌ها به کاربران اجازه می‌دهند که به سادگی
"pacman -S <نام گروه>" را اجرا کنند تا بسته‌های زیادی را دریافت کنند.

\subsubsection{blackarch}
گروه blackarch گروه پایه است که همه بسته‌ها باید به آن تعلق داشته باشند. این به
کاربران اجازه می‌دهد که همه بسته‌ها را با آسانی نصب کنند.

چه چیزی باید اینجا باشد: همه چیز.

\subsubsection{blackarch-anti-forensic}
بسته‌هایی که برای مقابله با فعالیت‌های پزشکی قانونی استفاده می‌شوند،
شامل رمزگذاری، استگانوگرافی، و هر چیزی که فایل‌ها/ویژگی‌های فایل را تغییر می‌دهد.
این همه شامل ابزارهایی برای کار با هر چیزی به طور کلی است که تغییرات در سیستم
به منظور پنهان کردن اطلاعات ایجاد می‌کند.

نمونه‌ها: luks, TrueCrypt, Timestomp, dd, ropeadope, secure-delete

\subsubsection{blackarch-automation}
بسته‌هایی که برای اتوماسیون ابزار یا گردش کار استفاده می‌شوند.

نمونه‌ها: blueranger, tiger, wiffy

\subsubsection{blackarch-backdoor}
بسته‌هایی که از سیستم‌های آسیب‌پذیر سوءاستفاده کرده یا بک‌دور باز می‌کنند.

نمونه‌ها: backdoor-factory, rrs, weevely

\subsubsection{blackarch-binary}
بسته‌هایی که روی فایل‌های باینری به شکلی عمل می‌کنند.

نمونه‌ها: binwally, packerid

\subsubsection{blackarch-bluetooth}
بسته‌هایی که از هر چیزی مربوط به استاندارد بلوتوث (802.15.1) سوءاستفاده می‌کنند.

نمونه‌ها: ubertooth, tbear, redfang

\subsubsection{blackarch-code-audit}
بسته‌هایی که کد منبع موجود را برای تجزیه و تحلیل آسیب‌پذیری ممیزی می‌کنند.

نمونه‌ها: flawfinder, pscan

\subsubsection{blackarch-cracker}
بسته‌های استفاده شده برای شکستن توابع رمزنگاری، یعنی هش‌ها.

نمونه‌ها: hashcat, john, crunch

\subsubsection{blackarch-crypto}
بسته‌هایی که با رمزنگاری کار می‌کنند، به استثنای شکستن.

نمونه‌ها: ciphertest, xortool, sbd

\subsubsection{blackarch-database}
بسته‌هایی که شامل سوءاستفاده از پایگاه داده در هر سطحی هستند.

نمونه‌ها: metacoretex, blindsql

\subsubsection{blackarch-debugger}
بسته‌هایی که به کاربر اجازه می‌دهند ببیند که یک برنامه خاص در زمان واقعی "چه کاری" انجام می‌دهد.

نمونه‌ها: radare2, shellnoob

\subsubsection{blackarch-decompiler}
بسته‌هایی که سعی می‌کنند یک برنامه کامپایل شده را به کد منبع معکوس کنند.

نمونه‌ها: flasm, jd-gui

\subsubsection{blackarch-defensive}
بسته‌هایی که برای محافظت از کاربر در برابر بدافزار و حملات سایر کاربران استفاده می‌شوند.

نمونه‌ها: arpon, chkrootkit, sniffjoke

\subsubsection{blackarch-disassembler}
این شبیه به blackarch-decompiler است، و احتمالاً برنامه‌های زیادی خواهند بود
که در هر دو دسته قرار می‌گیرند، با این حال این بسته‌ها خروجی اسمبلی تولید می‌کنند
به جای کد منبع خام.

نمونه‌ها: inguma, radare2

\subsubsection{blackarch-dos}
بسته‌هایی که از حملات DoS (انکار سرویس) استفاده می‌کنند.

نمونه‌ها: 42zip, nkiller2

\subsubsection{blackarch-drone}
بسته‌هایی که برای مدیریت پهپادهای
مهندسی شده فیزیکی استفاده می‌شوند.

نمونه‌ها: meshdeck, skyjack

\subsubsection{blackarch-exploitation}
بسته‌هایی که از آسیب‌پذیری‌ها در برنامه‌ها یا سرویس‌های دیگر سوءاستفاده می‌کنند.

نمونه‌ها: armitage, metasploit, zarp

\subsubsection{blackarch-fingerprint}
بسته‌هایی که از تجهیزات بیومتریک اثر انگشت سوءاستفاده می‌کنند.

نمونه‌ها: dns-map, p0f, httprint

\subsubsection{blackarch-firmware}
بسته‌هایی که از آسیب‌پذیری‌ها در فریم‌ویر سوءاستفاده می‌کنند

نمونه‌ها: هنوز هیچ‌کدام، در اسرع وقت اصلاح شود.

\subsubsection{blackarch-forensic}
بسته‌هایی که برای یافتن داده‌ها روی دیسک‌های فیزیکی یا حافظه تعبیه شده استفاده می‌شوند.

نمونه‌ها: aesfix, nfex, wyd

\subsubsection{blackarch-fuzzer}
بسته‌هایی که از اصل تست فاز استفاده می‌کنند، یعنی
"پرتاب" ورودی‌های تصادفی به سوژه برای دیدن اینکه چه اتفاقی می‌افتد.

نمونه‌ها: msf, mdk3, wfuzz

\subsubsection{blackarch-hardware}
بسته‌هایی که از هر چیزی مربوط به
سخت‌افزار فیزیکی سوءاستفاده یا مدیریت می‌کنند.

نمونه‌ها: arduino, smali

\subsubsection{blackarch-honeypot}
بسته‌هایی که به عنوان "عسل‌دان" عمل می‌کنند، یعنی برنامه‌هایی که به نظر
سرویس‌های آسیب‌پذیر هستند که برای جذب هکرها به دام استفاده می‌شوند.

نمونه‌ها: artillery, bluepot, wifi-honey

\subsubsection{blackarch-keylogger}
بسته‌هایی که ضربات کلید را در سیستم دیگری ضبط و نگهداری می‌کنند.

نمونه‌ها: هنوز هیچ‌کدام، در اسرع وقت اصلاح شود.

\subsubsection{blackarch-malware}
بسته‌هایی که به عنوان هر نوع نرم‌افزار مخرب یا
تشخیص بدافزار محسوب می‌شوند.

نمونه‌ها: malwaredetect, peepdf, yara

\subsubsection{blackarch-misc}
بسته‌هایی که به طور خاص در هیچ دسته‌ای قرار نمی‌گیرند.

نمونه‌ها: oh-my-zsh-git, winexe, stompy

\subsubsection{blackarch-mobile}
بسته‌هایی که پلتفرم‌های موبایل را دستکاری می‌کنند.

نمونه‌ها: android-sdk-platform-tools, android-udev-rules

\subsubsection{blackarch-networking}
بسته‌هایی که شامل شبکه‌سازی IP هستند.

نمونه‌ها: arptools, dnsdiag, impacket

\subsubsection{blackarch-nfc}
بسته‌هایی که از nfc (ارتباطات میدان نزدیک) استفاده می‌کنند.

نمونه‌ها: nfcutils

\subsubsection{blackarch-packer}
بسته‌هایی که روی پکرها عمل می‌کنند یا شامل آن‌ها هستند.

\textit{پکرها برنامه‌هایی هستند که بدافزار را در اجرایی‌های دیگر جاسازی می‌کنند.}

نمونه‌ها: packerid

\subsubsection{blackarch-proxy}
بسته‌هایی که به عنوان پراکسی عمل می‌کنند، یعنی ترافیک را
از طریق گره دیگری در اینترنت هدایت می‌کنند.

نمونه‌ها: burpsuite, ratproxy, sslnuke

\subsubsection{blackarch-recon}
بسته‌هایی که به طور فعال به دنبال آسیب‌پذیری‌های قابل بهره‌برداری در
طبیعت می‌گردند. بیشتر یک گروه چتری برای بسته‌های مشابه.

نمونه‌ها: canri, dnsrecon, netmask

\subsubsection{blackarch-reversing}
این یک گروه چتری برای هر دیکامپایلر،
دیس‌اسمبلر یا هر برنامه مشابه است.

نمونه‌ها: capstone, radare2, zerowine

\subsubsection{blackarch-scanner}
بسته‌هایی که سیستم‌های انتخاب شده را برای آسیب‌پذیری‌ها اسکن می‌کنند.

نمونه‌ها: scanssh, tiger, zmap

\subsubsection{blackarch-sniffer}
بسته‌هایی که شامل تجزیه و تحلیل ترافیک شبکه هستند.

نمونه‌ها: hexinject, pytactle, xspy

\subsubsection{blackarch-social}
بسته‌هایی که در درجه اول به سایت‌های شبکه‌های اجتماعی حمله می‌کنند.

نمونه‌ها: jigsaw, websploit

\subsubsection{blackarch-spoof}
بسته‌هایی که سعی می‌کنند مهاجم را جعل کنند به طوری که
مهاجم برای قربانی به عنوان یک مهاجم ظاهر نشود.

نمونه‌ها: arpoison, lans, netcommander

\subsubsection{blackarch-threat-model}
بسته‌هایی که برای گزارش‌دهی/ضبط
مدل تهدید ترسیم شده در یک سناریوی خاص استفاده می‌شوند.

نمونه‌ها: magictree

\subsubsection{blackarch-tunnel}
بسته‌هایی که برای تونل زدن ترافیک شبکه در یک
شبکه داده شده استفاده می‌شوند.

نمونه‌ها: ctunnel, iodine, ptunnel

\subsubsection{blackarch-unpacker}
بسته‌هایی که برای استخراج بدافزار از پیش بسته‌بندی شده از یک
اجرایی استفاده می‌شوند.

نمونه‌ها: js-beautify

\subsubsection{blackarch-voip}
بسته‌هایی که روی برنامه‌ها و پروتکل‌های voip عمل می‌کنند.

نمونه‌ها: iaxflood, rtp-flood, teardown

\subsubsection{blackarch-webapp}
بسته‌هایی که روی برنامه‌های مواجه با اینترنت عمل می‌کنند.

نمونه‌ها: metoscan, whatweb, zaproxy

\subsubsection{blackarch-windows}
این گروه برای هر بسته بومی ویندوز است که از طریق wine اجرا می‌شود.

نمونه‌ها: 3proxy-win32, pwdump, winexe

\subsubsection{blackarch-wireless}
بسته‌هایی که روی شبکه‌های بی‌سیم در هر سطحی عمل می‌کنند.

نمونه‌ها: airpwn, mdk3, wiffy

\section{ساختار مخزن}
می‌توانید مخزن اصلی git BlackArch را اینجا پیدا کنید:
\href{https://github.com/BlackArch/blackarch}{https://github.com/BlackArch/blackarch}.
همچنین چندین مخزن ثانویه اینجا وجود دارد:
\href{https://github.com/BlackArch}{https://github.com/BlackArch}.

در مخزن اصلی git، سه دایرکتوری مهم وجود دارد:

\begin{itemize}
\item docs - مستندات.
\item packages - فایل‌های PKGBUILD.
\item scripts - اسکریپت‌های کوچک مفید.
\end{itemize}

\subsection{اسکریپت‌ها}
در اینجا مرجعی برای اسکریپت‌ها در دایرکتوری \verb|scripts/| است:

\begin{itemize}
\item baaur - به زودی، این بسته‌ها را به AUR آپلود خواهد کرد.
\item babuild - یک بسته بسازید.
\item bachroot - یک chroot برای تست مدیریت کنید.
\item baclean - فایل‌های قدیمی .pkg.tar.xz را از مخزن بسته پاک کنید.
\item baconflict - به زودی این جایگزین \verb|scripts/conflicts| خواهد شد.
\item bad-files - فایل‌های بد در بسته‌های ساخته شده پیدا کنید.
\item balock - قفل مخزن بسته را بدست آورید یا آزاد کنید.
\item banotify - IRC را در مورد push های بسته آگاه کنید.
\item barelease - بسته‌ها را به مخزن بسته منتشر کنید.
\item baright - اطلاعات کپی‌رایت BlackArch را چاپ کنید.
\item basign - بسته‌ها را امضا کنید.
\item basign-key - یک کلید را امضا کنید.
\item blackman - این تقریباً مانند pacman رفتار می‌کند اما از git می‌سازد (نباید با
    Blackman nrz اشتباه گرفته شود).
\item check-groups - گروه‌ها را بررسی کنید.
\item checkpkgs - بسته‌ها را برای خطا بررسی کنید.
\item conflicts - تداخل فایل‌ها را بررسی کنید.
\item dbmod - یک پایگاه داده بسته را تغییر دهید.
\item depth-list - یک لیست مرتب شده بر اساس عمق وابستگی ایجاد کنید.
\item deptree - یک درخت وابستگی ایجاد کنید، فقط بسته‌های ارائه شده توسط blackarch را لیست کنید.
\item get-blackarch-deps - یک لیست از وابستگی‌های blackarch برای یک بسته بگیرید.
\item get-official - بسته‌های رسمی را برای انتشار دریافت کنید.
\item list-loose-packages - بسته‌هایی را لیست کنید که در گروه‌ها نیستند و
    وابستگی برای بسته‌های دیگر نیستند.
\item list-needed - وابستگی‌های گمشده را لیست کنید.
\item list-removed - بسته‌هایی را لیست کنید که در مخزن بسته هستند اما در git نیستند.
\item list-tools - ابزارها را لیست کنید.
\item outdated - به دنبال بسته‌هایی بگردید که در مخزن بسته منقضی شده‌اند
    نسبت به مخزن git.
\item pkgmod - یک بسته ساخت را تغییر دهید.
\item pkgrel - pkgrel را در یک بسته افزایش دهید.
\item prep - سبک یک فایل PKGBUILD را پاک کنید و خطاها را پیدا کنید.
\item sitesync - بین یک کپی محلی از مخزن بسته و یک کپی راه دور همگام‌سازی کنید.
\item size-hunt - به دنبال بسته‌های بزرگ بگردید.
\item source-backup - فایل‌های منبع بسته را پشتیبان‌گیری کنید.
\end{itemize}

\section{مشارکت در مخزن}
این بخش به شما نشان می‌دهد که چگونه در پروژه BlackArch Linux مشارکت کنید. ما
درخواست‌های pull از هر اندازه‌ای را می‌پذیریم، از اصلاحات ریز تایپی تا بسته‌های جدید.\\برای
کمک، پیشنهادات، یا سؤالات با ما تماس بگیرید.
\\\\
همه برای مشارکت خوش آمدند. همه مشارکت‌ها قدردانی می‌شود.

\subsection{آموزش‌های مورد نیاز}
لطفاً قبل از مشارکت آموزش‌های زیر را بخوانید:
\begin{itemize}
\item
\href{https://wiki.archlinux.org/index.php/Arch\_Packaging\_Standards)}{استانداردهای
بسته‌بندی Arch}
\item \href{https://wiki.archlinux.org/index.php/Creating\_Packages}{ایجاد
بسته‌ها}
\item \href{https://wiki.archlinux.org/index.php/PKGBUILD}{PKGBUILD}
\item \href{https://wiki.archlinux.org/index.php/Makepkg}{Makepkg}
\end{itemize}

\subsection{مراحل مشارکت}
برای ارسال تغییرات خود به پروژه BlackArchLinux، این
مراحل را دنبال کنید:
\begin{enumerate}
\item مخزن را از
\url{https://github.com/BlackArch/blackarch} فورک کنید
\item فایل‌های لازم را تغییر دهید، (مثل PKGBUILD، فایل‌های .patch و غیره).
\item تغییرات خود را کامیت کنید.
\item تغییرات خود را پوش کنید.
\item از ما بخواهید تغییرات شما را ادغام کنیم، ترجیحاً از طریق یک pull request.
\end{enumerate}

\subsection{مثال}
مثال زیر ارسال یک بسته جدید به پروژه BlackArch را
نشان می‌دهد. ما از \href{https://github.com/Jguer/yay}{yay}
(می‌توانید از pacaur نیز استفاده کنید) برای دریافت یک فایل PKGBUILD از پیش موجود برای
\textbf{nfsshell} از \href{https://aur.archlinux.org/}{AUR} استفاده می‌کنیم و آن را
طبق نیازهای ما تنظیم می‌کنیم.

\subsubsection{دریافت PKGBUILD}
فایل \textit{PKGBUILD} را با استفاده از yay یا pacaur دریافت کنید:
\begin{lstlisting}
  user@blackarchlinux $ yay -G nfsshell
  ==> Download nfsshell sources
  x LICENSE
  x PKGBUILD
  x gcc.patch
  user@blackarchlinux $ cd nfsshell/
\end{lstlisting}

\subsubsection{پاک‌سازی PKGBUILD}
فایل \textit{PKGBUILD} را پاک کنید و مقداری وقت صرفه‌جویی کنید:
\begin{lstlisting}
  user@blackarchlinux nfsshell $ ./blackarch/scripts/prep PKGBUILD
  cleaning 'PKGBUILD'...
  expanding tabs...
  removing vim modeline...
  removing id comment...
  removing contributor and maintainer comments...
  squeezing extra blank lines...
  removing '|| return'...
  removing leading blank line...
  removing $pkgname...
  removing trailing whitespace...
\end{lstlisting}

\subsubsection{تنظیم PKGBUILD}
فایل \textit{PKGBUILD} را تنظیم کنید:
\begin{lstlisting}
  user@blackarchlinux nfsshell $ vi PKGBUILD
\end{lstlisting}

\subsubsection{ساخت بسته}
بسته را بسازید:
\begin{lstlisting}user@blackarchlinux nfsshell $ makepkg -sf
==> Making package: nfsshell 19980519-1 (Mon Dec  2 17:23:51 CET 2013)
==> Checking runtime dependencies...
==> Checking buildtime dependencies...
==> Retrieving sources...
-> Downloading nfsshell.tar.gz...
% Total    % Received % Xferd  Average Speed   Time    Time     Time
CurrentDload  Upload   Total   Spent    Left  Speed100 29213  100 29213    0
0  48150      0 --:--:-- --:--:-- --:--:-- 48206
-> Found gcc.patch
-> Found LICENSE
...
<lots of build process and compiler output here>
...
==> Leaving fakeroot environment.
==> Finished making: nfsshell 19980519-1 (Mon Dec  2 17:23:53 CET 2013)
\end{lstlisting}

\subsubsection{نصب و تست بسته}
بسته را نصب و تست کنید:
\begin{lstlisting}
  user@blackarchlinux nfsshell $ pacman -U nfsshell-19980519-1-x86_64.pkg.tar.xz
  user@blackarchlinux nfsshell $ nfsshell # test it
\end{lstlisting}

\subsubsection{اضافه، کامیت و پوش بسته}
بسته را اضافه، کامیت و پوش کنید
\begin{lstlisting}user@blackarchlinux nfsshell $ cd /blackarchlinux/packages
user@blackarchlinux ~/blackarchlinux/packages $ mv ~/nfsshell .
user@blackarchlinux ~/blackarchlinux/packages $ git commit -am nfsshell && git push
\end{lstlisting}

\subsubsection{ایجاد pull request}
یک pull request در \href{https://github.com/}{github.com} ایجاد کنید
\begin{lstlisting}
  firefox https://github.com/<contributor>/blackarchlinux
\end{lstlisting}

\subsubsection{اضافه کردن یک remote برای upstream}
کار هوشمندانه‌ای که باید انجام دهید اگر روی upstream و fork کار می‌کنید این است که fork خودتان را pull کنید و مخزن اصلی ba را به عنوان یک remote اضافه کنید
\begin{lstlisting}
  user@blackarchlinux ~/blackarchlinux $ git remote -v
  origin <the url of your fork> (fetch)
  origin <the url of your fork> (push)
  user@blackarchlinux ~/blackarchlinux $ git remote add upstream https://github.com/blackarch/blackarch
  user@blackarchlinux ~/blackarchlinux $ git remote -v
  origin <the url of your fork> (fetch)
  origin <the url of your fork> (push)
  upstream https://github.com/blackarch/blackarch (fetch)
  upstream https://github.com/blackarch/blackarch (push)
\end{lstlisting}

به طور پیش‌فرض، git باید مستقیماً به origin پوش کند، اما مطمئن شوید که تنظیمات git شما
به درستی پیکربندی شده است. این مشکلی نخواهد بود مگر اینکه حقوق کامیت داشته باشید چرا که
بدون آن‌ها نمی‌توانید به upstream پوش کنید.

اگر توانایی کامیت دارید، ممکن است با استفاده از
\textit{git@github.com:blackarch/blackarch.git} موفق‌تر باشید اما این به شما بستگی دارد.

\subsection{درخواست‌ها}
\begin{enumerate}
\item نظرات \textbf{Maintainer} یا \textbf{Contributor} را به
فایل‌های \textit{PKGBUILD} اضافه نکنید. نام‌های maintainer و contributor را به بخش
AUTHORS راهنمای BlackArch اضافه کنید.
\item برای حفظ سازگاری، لطفاً سبک کلی سایر
فایل‌های \textit{PKGBUILD} در مخزن را دنبال کنید و از تورفتگی دو فاصله استفاده کنید.
\end{enumerate}

\subsection{نکات کلی}
\href{http://wiki.archlinux.org/index.php/Namcap}{namcap} می‌تواند بسته‌ها را برای
خطا بررسی کند.

%------------------%
%  Chapter 4       %
%------------------%

\chapter{راهنمای ابزارها}
به زودی...

\section{به زودی}
به زودی...

%%% APPENDIX %%%
\appendix
%%%%%%%%%%%%%%%%%%%%%%%%%%%%%%%%%%%%%%%%%%%%%%%%%%%%%%%%%%%%%%%%%%%%%%%%%%%%%%%%
%                                                                              %
% BlackArch Linux قالب پیوست                                                  %
%                                                                              %
%%%%%%%%%%%%%%%%%%%%%%%%%%%%%%%%%%%%%%%%%%%%%%%%%%%%%%%%%%%%%%%%%%%%%%%%%%%%%%%%

\appendix

\chapter{پیوست}

\section{سوالات متداول}

\section{نویسندگان}
افراد زیر مستقیماً به BlackArch کمک کرده‌اند:
\begin{itemize}
\item Tyler Bennnett (tylerb@trix2voip.com)
\item fnord0 (fnord0@riseup.net)
\item nrz (nrz@nullsecurity.net)
\item Ellis Kenyo (elken.tdos@gmail.com)
\item CaledoniaProject (the.warl0ck.1989@gmail.com)
\item sudokode (sudokode@gmail.com)
\item Valentin Churavy (v.churavy@gmail.com)
\item Boy Sandy Gladies Arriezona (reno.esper@gmail.com)
\item Mathias Nyman
\item Johannes Löthberg (demizide@gmail.com)
\item Thiago da Silva Teixeira (teixeira.zeus@gmail.com)
\end{itemize}
افراد زیر مستقیماً در ArchPwn کمک کرده‌اند،
و به BlackArch پیوسته‌اند:
\begin{itemize}
\item Francesco Piccinno (stack.box@gmail.com)
\item jensp (jens@jenux.homelinux.org)
\item Valentin Churavy (v.churavy@gmail.com)
\end{itemize}
کد ساخت را از افراد زیر گرفته‌ایم:
\begin{itemize}
\item  3ED (krzysztof1987@gmail.com)
\item  AUR Perl (aurperl@juster.info)
\item  Aaron Griffin (aaron@archlinux.org)
\item  Abakus (java5@arcor.de)
\item  Adam Wolk (netprobe@gmail.com)
\item  Aleix Pol (aleixpol@kde.org)
\item  Aleshus (aleshusi@gmail.com)
\item  Alessandro Pazzaglia (jackdroido@gmail.com)
\item  Alessandro Sagratini (ale\_sagra@hotmail.com)
\item  Alex Cartwright (alexc223@googlemail.com)
\item  Alexander De Sousa (archaur.xandy21@spamgourmet.com)
\item  Alexander Rødseth (rodseth@gmail.com)
\item  Allan McRae (allan@archlinux.org)
\item  AmaN (gabroo.punjab.da@gmail.com)
\item  Andre Klitzing (aklitzing@online.de)
\item  Andrea Scarpino (andrea@archlinux.org)
\item  Andreas Schönfelder (passtschu@freenet.de)
\item  Andrej Gelenberg (andrej.gelenberg@udo.edu)
\item  Angel Velasquez (angvp@archlinux.org)
\item  Antoine Lubineau (antoine@lubignon.info)
\item  Anton Bazhenov (anton.bazhenov@gmail.com)
\item  Arkham (arkham@archlinux.us)
\item  Arthur Danskin (arthurdanskin@gmail.com)
\item  Balda (balda@balda.ch)
\item  Balló György (ballogyor+arch@gmail.com)
\item  Bartek Piotrowski (barthalion@gmail.com)
\item  Bartosz Feński (fenio@debian.org)
\item  Bartłomiej Piotrowski (nospam@bpiotrowski.pl)
\item  Bogdan Szczurek (thebodzio@gmail.com)
\item  Brad Fanella (bradfanella@archlinux.us)
\item  Brian Bidulock (bidulock@openss7.org)
\item  C Anthony Risinger (anthony@xtfx.me)
\item  CRT (crt.011@gmail.com)
\item  Can Celasun (dcelasun@gmail.com)
\item  Chaniyth (chaniyth@yahoo.com)
\item  Chris Brannon (cmbrannon79@gmail.com)
\item  Chris Giles (Chris.G.27@gmail.com) \& daschu117
\item  Christoph Siegenthaler (csi@gmx.ch)
\item  Christoph Zeiler (archNOSPAM@moonblade.org)
\item  Clément DEMOULINS (clement@archivel.fr)
\item  Corrado Primier (bardo@aur.archlinux.org)
\item  Daenyth (Daenyth+Arch@gmail.com)
\item  Dale Blount (dale@archlinux.org)
\item  Damir Perisa (damir.perisa@bluewin.ch)
\item  Dan Fuhry (dan@fuhry.us)
\item  Dan Serban (dserban01@yahoo.com)
\item  Daniel A. Campoverde Carrión
\item  Daniel Golle
\item  Daniel Griffiths (ghost1227@archlinux.us)
\item  Daniel J Griffiths (ghost1227@archlinux.us)
\item  Daniel Micay (danielmicay@gmail.com)
\item  Dave Reisner (dreisner@archlinux.org)
\item  Dawid Wrobel (cromo@klej.net)
\item  Devaev Maxim (mdevaev@gmail.com)
\item  Devin Cofer (ranguvar@archlinux.us)
\item  DigitalPathogen (aur@InfoSecResearchLabs.co.uk)
\item  DigitalPathogen (aur@digitalpathogen.co.uk)
\item  Dmitry A. Ilyashevich (dmitry.ilyashevich@gmail.com)
\item  Dominik Heidler (dheidler@gmail.com)
\item  DrZaius (lou@fakeoutdoorsman.com)
\item  Ebubekir KARUL (ebubekirkarul@yandex.com)
\item  Eduard "bekks" Warkentin (eduard.warkentin@gmail.com)
\item  Elmo Todurov (todurov@gmail.com)
\item  Emmanuel Gil Peyrot (linkmauve@linkmauve.fr)
\item  Eric Belanger (eric@archlinux.org)
\item  Ermak (ermak@email.it)
\item  Evangelos Foutras (evangelos@foutrelis.com)
\item  Fabian Melters (melters@gmail.com)
\item  Fabiano Furtado (fusca14@gmail.com)
\item  Federico Quagliata (ntp@quaqo.org)
\item  Firmicus (francois.archlinux@org)
\item  Florian Pritz (bluewind@jabber.ccc.de)
\item  Florian Pritz (flo@xinu.at)
\item  Francesco Piccinno (stack.box@gmail.com)
\item  François Charette (francois@archlinux.org)
\item  Gaetan Bisson (bisson@archlinux.org)
\item  Geoffroy Carrier (geoffroy.carrier@koon.fr)
\item  Georg Grabler (STiAT)
\item  George Hilliard (gh403@msstate.edu)
\item  Gerardo Exequiel Pozzi (vmlinuz386@yahoo.com.ar)
\item  Gilles CHAUVIN (gcnweb@gmail.com)
\item  Giovanni Scafora (giovanni@archlinux.org)
\item  Gordin (9ordin@gmail.com)
\item  Guillaume ALAUX (guillaume@archlinux.org)
\item  Guillermo Ramos (0xwille@gmail.com)
\item  Gustavo Alvarez (sl1pkn07@gmail.com)
\item  Hugo Doria (hugo@archlinux.org)
\item  Hyacinthe Cartiaux (hyacinthe.cartiaux@free.fr)
\item  James Fryman (jfryman@gmail.com)
\item  Jan "heftig" Steffens (jan.steffens@gmail.com)
\item  Jan de Groot (jgc@archlinux.org)
\item  Jaroslav Lichtblau (dragonlord@aur.archlinux.org)
\item  Jaroslaw Swierczynski (swiergot@aur.archlinux.org)
\item  Jason Chu (jason@archlinux.org)
\item  Jason R Begley (jayray@digitalgoat.com)
\item  Jason Rodriguez
\item  Jason St. John (jstjohn@purdue.edu)
\item  Jawmare (victor2008@gmail.com)
\item  Jeff Mickey (jeff@archlinux.org)
\item  Jens Pranaitis (jens@chaox.net)
\item  Jens Pranaitis (jens@jenux.homelinux.org)
\item  Jinx (jinxware@gmail.com)
\item  John D Jones III (jnbek1972@gmail.com)
\item  John Proctor (jproctor@prium.net)
\item  Jon Bergli Heier (snakebite@jvnv.net)
\item  Jonas Heinrich
\item  Jonathan Steel (jsteel@aur.archlinux.org)
\item  Joris Steyn (jorissteyn@gmail.com)
\item  Josh VanderLinden (arch@cloudlery.com)
\item  Jozef Riha (jose1711@gmail.com)
\item  Judd Vinet (jvinet@zeroflux.org)
\item  Juergen Hoetzel (jason@archlinux.org)
\item  Juergen Hoetzel (juergen@archlinux.org)
\item  Justin Davis (jrcd83@gmail.com)
\item  Kaiting Chen (kaitocracy@gmail.com)
\item  Kaos
\item  Kevin Piche (kevin@archlinux.org)
\item  Kory Woods (kory@virlo.net)
\item  Kyle Keen (keenerd@gmail.com)
\item  Larry Hajali (larryhaja@gmail.com)
\item  LeCrayonVert
\item  Le\_suisse (lesuisse.dev+aur@gmail.com)
\item  Lekensteyn (lekensteyn@gmail.com)
\item  Limao Luo (luolimao+AUR@gmail.com)
\item  Lucien Immink
\item  Lukas Fleischer (archlinux@cryptocrack.de)
\item  Manolis Tzanidakis
\item  Marcin "avalan" Falkiewicz (avalatron@gmail.com)
\item  Mariano Verdu (verdumariano@gmail.com)
\item  Marti Raudsepp (marti@juffo.org)
\item  MatToufoutu (mattoufootu@gmail.com)
\item  Matthew Sharpe (matt.sharpe@gmail.com)
\item  Mauro Andreolini (mauro.andreolini@unimore.it)
\item  Max Pray a.k.a. Synthead (synthead@gmail.com)
\item  Max Roder (maxroder@web.de)
\item  Maxwell Pray a.k.a. Synthead (synthead@gmail.com)
\item  Maxwell Pray a.k.a. Synthead (synthead1@gmail.com)
\item  Mech (tiago.bmp@gmail.com)
\item  Michael Düll (mail@akurei.me)
\item  Michael P (ptchinster@archlinux.us)
\item  Michal Krenek (mikos@sg1.cz)
\item  Michal Zalewski (lcamtuf@coredump.cx)
\item  Miguel Paolino (mpaolino@gmail.com)
\item  Miguel Revilla (yo@miguelrevilla.com)
\item  Mike Roberts (noodlesgc@gmail.com)
\item  Mike Sampson (mike@sambodata.com)
\item  Nassim Kacha (nassim.kacha@gmail.com)
\item  Nicolas Pouillard (nicolas.pouillard@gmail.com)
\item  Nicolas Pouillard https://nicolaspouillard.fr
\item  Niklas Schmuecker
\item  Oleander Reis (oleander@oleander.cc)
\item  Olivier Le Moal (mail@olivierlemoal.fr)
\item  Olivier Médoc "oliv" (o\_medoc@yahoo.fr)
\item  Pascal E. (archlinux@hardfalcon.net)
\item  Patrick Leslie Polzer (leslie.polzer@gmx.net)
\item  Paul Mattal (paul@archlinux.org)
\item  Paul Mattal (pjmattal@elys.com)
\item  Pengyu CHEN (cpy.prefers.you@gmail.com)
\item  Peter Wu (lekensteyn@gmail.com)
\item  Philipp 'TamCore' B. (philipp@tamcore.eu)
\item  Pierre Schmitz (pierre@archlinux.de)
\item  Pranay Kanwar (pranay.kanwar@gmail.com)
\item  Pranay Kanwar (warl0ck@metaeye.org)
\item  PyroPeter (googlemail@com.abi1789)
\item  PyroPeter (googlemail.com@abi1789)
\item  Ray Rashif (schiv@archlinux.org)
\item  Remi Gacogne
\item  Renan Fernandes (renan@kauamanga.com)
\item  Richard Murri (admin@richardmurri.com)
\item  Roberto Alsina (ralsina@kde.org)
\item  Robson Peixoto (robsonpeixoto@gmail.com)
\item  Roel Blaauwgeers (roel@ttys0.nl)
\item  Rorschach (r0rschach@lavabit.com)
\item  Ruben Schuller (shiml@orgizm.net)
\item  Rudy Matela (rudy@matela.com)
\item  Ryon Sherman (ryon.sherman@gmail.com)
\item  Sabart Otto \item  Seberm (seberm@gmail.com)
\item  SakalisC (chrissakalis@gmail.com)
\item  Sam Stuewe (halosghost@archlinux.info)
\item  SanskritFritz (SanskritFritz@gmail.com)
\item  Sarah Hay (sarahhay@mb.sympatico)
\item  Sebastian Benvenuti (sebastianbenvenuti@gmail.com)
\item  Sebastian Nowicki (sebnow@gmail.com)
\item  Sebastien Duquette (ekse.0x@gmail.com)
\item  Sebastien LEDUC (sebastien@sleduc.fr)
\item  Sebastien Leduc (sebastien@sleduc.fr)
\item  Sergej Pupykin (pupykin.s+arch@gmail.com)
\item  Sergio Rubio (rubiojr@biondofu.net)
\item  Sheng Yu (magicfish1990@gmail.com)
\item  Simon Busch (morphis@gravedo.de)
\item  Simon Legner (Simon.Legner@gmail.com)
\item  Sirat18 (aur@sirat18.de)
\item  SpepS (dreamspepser@yahoo.it)
\item  Spider.007 (archPackage@spider007.net)
\item  Stefan Seering
\item  Stephane Travostino (stephane.travostino@gmail.com)
\item  Stéphane Gaudreault (stephane@archlinux.org)
\item  Sven Kauber (celeon@gmail.com)
\item  Sven Schulz (omee@archlinux.de)
\item  Sébastien Duquette (ekse.0x@gmail.com)
\item  Sébastien Luttringer (seblu@archlinux.org)
\item  TDY (tdy@gmx.com)
\item  Teemu Rytilahti (tpr@iki.fi)
\item  Testuser\_01
\item  Thanx (thanxm@gmail.com)
\item  Thayer Williams (thayer@archlinux.org)
\item  Thomas S Hatch (thatch45@gmail.com)
\item  Thorsten Töpper
\item  Tilmann Becker (tilmann.becker@freenet.de)
\item  Timothy Redaelli (timothy.redaelli@gmail.com)
\item  Timothée Ravier (tim@siosm.fr)
\item  Tino Reichardt
\item  Tobias Kieslich (tobias@justdreams.de)
\item  Tobias Powalowski (tpowa@archlinux.org)
\item  Tom K (tomk@runbox.com)
\item  Tom Newsom (Jeepster@gmx.co.uk)
\item  Tomas Lindquist Olsen (tomas@famolsen.dk)
\item  Travis Willard (travisw@wmpub.ca)
\item  Valentin Churavy (v.churavy@gmail.com)
\item  ViNS (gladiator@fastwebnet.it)
\item  Vlatko Kosturjak (kost@linux.hr)
\item  Wes Brown (wesbrown18@gmail.com)
\item  William Rea (sillywilly@gmail.com)
\item  Xavier Devlamynck (magicrhesus@ouranos.be)
\item  Xiao\item Long Chen (chenxiaolong@cxl.epac.to)
\item  aeolist (aeolist@hotmail.com)
\item  ality@pbrane.org
\item  astaroth (astaroth\_@web.de)
\item  bender02@archlinux.us
\item  billycongo (billycongo@gmail.com)
\item  bslackr (brendan@vastactive.com)
\item  cbreaker (cbreaker@tlen.pl)
\item  chimeracoder (dev@chimeracoder.net)
\item  damir (damir@archlinux.org)
\item  danitool
\item  darkapex (me@jailuthra.in)
\item  daronin
\item  dkaylor (dpkaylor@gmail.com)
\item  dobo (dobo90\_at\_gmail@com)
\item  dorphell (dorphell@archlinux.org)
\item  evr (evanroman@at.gmail)
\item  fnord0 (fnord0@riseup.net)
\item  fxbru (frxbru@gmail)
\item  hcar
\item  icarus (icarus.roaming@gmail.com)
\item  iceman (icemanf@gmail.com)
\item  kastor (kastor@fobos.org)
\item  kfgz (kfgz@interia.pl)
\item  linuxSEAT (linuxSEAT@gmail.com)
\item  m4xm4n (max@maxfierke.com)
\item  mar77i (mysatyre@gmail.com)
\item  marc0s (marc0s@fsfe.org)
\item  mickael9 (mickael9@gmail.com)
\item  nblock (nblock@archlinux.us)
\item  nofxx (x@nofxx.com)
\item  onny (onny@project
\item  pootzko (pootzko@gmail.com)
\item  revel (revel@muub.net)
\item  rich\_o (rich\_o@lavabit.com)
\item  s1gma (s1gma@mindslicer.com)
\item  sandman (r.coded@gmail.com)
\item  sebikul (sebikul@gmail.com)
\item  sh0 (mee@sh0.org)
\item  shild (sxp@bk.ru)
\item  simo (simo@archlinux.org)
\item  snuo
\item  sudokode (sudokode@gmail.com)
\item  tobias (tobias@archlinux.org)
\item  trashstar (trash@ps3zone.org)
\item  unexist (unexist@subforge.org)
\item  untitled (rnd0x00@gmail.com)
\item  virtuemood (virtue@archlinux.us)
\item  wido (widomaker2k7@gmail.com)
\item  wodim (neikokz@gmail.com)
\item  yannsen (ynnsen@gmail.com)
\end{itemize}

\end{document}

%%% EOF %%%
